\section{Microcontroller}
	\subsection{Specs}
	
	\subsection{The Clock}
		\begin{itemize}
			\item Clock pulses generated by an Oscillator
			\item Square wave (or close to)
			\item EVBPlus uses and external 8MHz clock
			\item Internally the clock is divided by 4, resulting in an E-Clock of 2MHz
			\item 1 Clock cycle takes 1/2MHZ = 500nSec
			\item Internal CPU operations are synched to clock pulses
		\end{itemize}
		
	\subsection{Memory}
		\begin{itemize}
			\item Generally two types, ROM and RAM
			\item Accessed via a bus system
		\end{itemize}

	\subsection{I/O Registers}
		\begin{itemize}
			\item Used for accessing I/O
			\item Write to registers to configure or to send out data
		\end{itemize}
	
	\subsection{CPU}
		\subsubsection{Control Sequencer}
			\begin{itemize}
				\item Used to co-ordinate data transfers to and from the CPU
				\item Logic governed by MicroPrograms, small sets of instruction built into the CPU
			\end{itemize}
		
		\subsubsection{Instruction Decoder}
			\begin{itemize}
				\item Receives opcodes and interprets them
				\item Uses MicroPrograms to tell the ALU what actions to perform
			\end{itemize}
		
		\subsubsection{ALU}
			\begin{itemize}
				\item The Arithmetic Logic Unit carries out logic and logic operations on the operands
				\item 68HC11 has support for addition, subtraction, multiplication and division commands
			\end{itemize}
		
		\subsubsection{Address Bus}
			\begin{itemize}
				\item A bus is a topology or circuit in which all devices are directly connected to a line
and signals pass through all the devices. Each device has a unique identity so it can recognise the signals intended for it.
				\item The address bus of the 68HC11 is 16 parallel wires attached to a number of devices including:
CPU, internal and external memory etc.
			\end{itemize}
		
		\subsubsection{Data Bus}
			\begin{itemize}
				\item In the 68HC11 the Databus is an 8 bit bus which data can be sent along
				\item Used with the Address Bus. Address bus identifies the location to send or receive data.
The Data bus carries the data to be sent or received.
			\end{itemize}
		
		\subsubsection{CPU Registers}
			\begin{itemize}
			\item{Accumulators}
				\begin{itemize}
					\item Where the results from arithmetic operations are stored
					\item There are two 8-bit accumulators (A and B) in the 68HC11.
They can be combined to create a third 16-bit register (D).
				\end{itemize}

			\item{Index Registers}
				\begin{itemize}
					\item Used to point to locations in memory
					\item Can also be use for temporary data storage and arithmetic operations too
					\item Two 16-bit registers (X and Y) on the 68HC11
				\end{itemize}
			
			\item{Program Counter}
				\begin{itemize}
					\item Stores the location of the next instruction to be executed
					\item 16-bit register on the 68HC11
				\end{itemize}
			
			\item{Stack Pointer}
				\begin{itemize}
					\item Points to the top of the memory stack
					\item 16-bit register on 68HC11
				\end{itemize}

			\item{Condition Code Register}
				\begin{itemize}
					\item Used to record the result of some arithmetic operations
					\item 8-bit register on 68HC11
				\end{itemize}
			\end{itemize}


