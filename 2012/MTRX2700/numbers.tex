\section{Numbers}
	\subsection{Binary}
		\subsubsection{Definitions}
			\begin{description}
				\item[Bit] Binary Digit
				\item[4 bits] A nibble
				\item[8 bits] A byte
				\item[16 bits] A word
			\end{description}
			
		\subsubsection{Signs}
			\begin{description}
				\item[Unsigned 1 byte integer] Range from 0 - 255
				\item[Signed 1 byte integer] Range from -128 to 127
			\end{description}
			
		\subsection{Floating Point Numbers}
			The notation of a floating point number is $p*b^q$.
			\begin{description}
				\item[p] Mantissa or significand
				\item[b] Base
				\item[q] Exponent
			\end{description}
		
	\subsection{Hexadecimal}
		Base 16, from 0 - F.
	
	\subsection{ASCII}
		Translates a number into a character.
		1 byte per character.
	
	\subsection{Binary operations}
		\subsubsection{AND}
			Binary operation which gives the bits which are set to 1 in both numbers.
		\subsubsection{OR}
			Binary operation which gives the bits which are set to 1 in either of the numbers.
		\subsubsection{XOR}
			Binary operation which gives the bits which are set to 1 in one and only one of the numbers.
